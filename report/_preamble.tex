\usepackage[utf8]{inputenc}
\usepackage[
backend=biber,
style=numeric,
sorting=ynt
]{biblatex}
\addbibresource{References.bib}

% Sideopsætning
\usepackage{geometry}  
% Håndtering af papirstørrelse og marginer
\geometry{a4paper, twoside}                 % Papirstørrelse
\geometry{top=2.5cm, bottom=2.5cm}		    % Øvre og nedre margin
\geometry{left=2.5cm, right=2.5cm}		    % Venstre og højre margin

\usepackage{multirow} 
\usepackage{graphicx}
\usepackage{booktabs}
\usepackage{babel}

\addbibresource{References.bib}
\usepackage{
    amsmath, % \begin{align}
    amssymb, % math symbols såsom \rightarrow 
    graphicx, % indsæt billeder
    wrapfig, % indsæt figurer ved siden af tekst
    float, % vælge placering af figures med [H]
    enumitem, % ændre label på enumeration
    fancyhdr, % header og footer
    colortbl, % farve tabellers celler
    tabularx, % mere kontrol over tabeller
    listings, % kodebokse
    hyperref, % embed links
    nameref, % referere til mere end bare label-number
    pdfpages,
} 

\usepackage[dvipsnames,table,longtable,x11namesx,xcdraw]{xcolor}
\definecolor{aublueclassic}{RGB}{0,61,115}
\definecolor{aubluedark}{RGB}{0,37,70}
\definecolor{aucyan}{RGB}{225,248,253}
%\definecolor{aucyan}{RGB}{55,160,203}
\definecolor{aucyandark}{RGB}{0,62,92}
\definecolor{lightGray}{RGB}{153,153,153}
\definecolor{darkGray}{RGB}{119,119,119}
\definecolor{khaki}{RGB}{240,230,140}
\definecolor{lavender}{RGB}{230,230,250}
\definecolor{OliveGreen}{rgb}{0.33, 0.42, 0.18}
\definecolor{Fuchsia}{rgb}{0.76, 0.33, 0.76}
\definecolor{NavyBlue}{rgb}{0.0, 0.0, 0.5}
\definecolor{LimeGreen}{rgb}{0.2, 0.8, 0.2}

\usepackage{tikz}
\usepackage{colortbl} 

% Sideopsætning
\pagestyle{fancy} % Bred side med header og footer
\fancyhead[LR]{} % Headeren er nu tom
\renewcommand{\headrulewidth}{0pt} % Ingen streg under headeren
\renewcommand{\footrulewidth}{0pt} % Ingen streg under footeren
\setcounter{tocdepth}{3} % Hvor meget skal med i tocs

% Tekstformattering
\setlength{\parindent}{0pt} % Ingen indryk
\linespread{1.2} % Linjeafstand
\setlength{\parskip}{1ex plus 0.5ex minus 0.2ex} % Et lille linjeskip i stedet for indryk

\newcommand{\dashline}{- - - - - - - - - - - - - - - - - - - - - - - - - - - - - - - - - - - - - - - - - - - - - - - - - - - - - - - - - - - -}
%\newcommand{\dashline}{-- \;\;-- \;\;-- \;\;-- \;\;-- \;\;-- \;\;-- \;\;-- \;\;-- \;\;-- \;\;-- \;\;-- \;\;-- \;\;-- \;\;-- \;\;-- \;\;-- \;\;-- \;\;-- \;\;-- \;\;-- \;\;-- \;\;-- \;\;-- \;\;-- \;\;-- \;\;-- \;\;-- \;\;-- \;\;-- \;\;-- \;\;-- \;\;--}


\hypersetup{
    colorlinks,
    linkcolor={black},
    citecolor={black},
    urlcolor={blue!80!black}
}


% Tabeller, afhængig af tabularx package
\newcolumntype{R}{>{\raggedleft\arraybackslash}X}
\newcolumntype{B}{>{\columncolor{mygray}\raggedright}X}
\newcolumntype{L}{>{\raggedright}X}
\newcolumntype{C}{>{\centering\arraybackslash}X}


% Matematik
\renewcommand{\mod}{\textbf{ mod }}
\newcommand{\mmod}{\text{mod }}
\renewcommand{\lg}{\text{lg}}
\renewcommand{\div}{\text{ div }}


% Farver
\definecolor{darkgreen}{HTML}{009900}
\definecolor{dkgreen}{HTML}{21892F}
\definecolor{myblue}{HTML}{3237C0}
\definecolor{mygrey}{rgb}{0.5,0.5,0.5}

%%%% TABEL BAGGRUNDSFARVER %%%% CREDIT INGRID
\definecolor{aublueclassic}{RGB}{0,61,115}
\definecolor{aubluedark}{RGB}{0,37,70}
\definecolor{aucyan}{RGB}{225,248,253}
%\definecolor{aucyan}{RGB}{55,160,203}
\definecolor{aucyandark}{RGB}{0,62,92}
\definecolor{lightGray}{RGB}{153,153,153}
\definecolor{darkGray}{RGB}{119,119,119}
\definecolor{khaki}{RGB}{240,230,140}
\definecolor{lavender}{RGB}{230,230,250}

% farvet tekst, afhængig af colorx package
\newcommand{\red}[1]{\textcolor{red}{#1}}
\newcommand{\grey}[1]{\textcolor{lightgray}{#1}}
\newcommand{\green}[1]{\textcolor{darkgreen}{#1}}

% farvet tekst til brug af at tydeligøre ændringer - afhængig af colorx package
\definecolor{addgreen}{RGB}{0,204,0}
\definecolor{modifyyellow}{RGB}{255,153,51}
\definecolor{removered}{RGB}{204,0,0}

\newcommand{\added}[1]{\textcolor{addgreen}{#1}}
\newcommand{\modified}[1]{\textcolor{modifyyellow}{#1}}
\newcommand{\removed}[1]{\textcolor{removered}{#1}}


% Til kodebokse, afhængig af listings package
\lstset{frame=single,
  language=Java,
  aboveskip=3mm,
  belowskip=3mm,
  showstringspaces=false,
  columns=flexible,
  basicstyle={\small\ttfamily},
  numbers=left, 
  numberstyle=\tiny\color{mygrey},
  keywordstyle=\color{myblue},
  commentstyle=\color{dkgreen},
  breaklines=true,
  breakatwhitespace=true,
  tabsize=3
}
